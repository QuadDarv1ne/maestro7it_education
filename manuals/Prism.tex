% !TEX program = xelatex
% NOTE: Для корректного Unicode используйте XeLaTeX (или LuaLaTeX).
\documentclass[12pt]{article}
\usepackage{fontspec}
\setmainfont{CMU Serif}
\usepackage[margin=1in]{geometry}
\usepackage{amsmath,amssymb,amsthm}
\usepackage{graphicx}
\usepackage{tikz-cd}
\usepackage{multicol}
\usepackage{hyperref}
\usepackage{listings}
\usepackage{xcolor}
\usepackage{fancyhdr}
\usepackage{titlesec}

% Настройка заголовков
\pagestyle{fancy}
\fancyhf{}
\fancyhead[L]{\leftmark}
\fancyhead[R]{\thepage}
\renewcommand{\headrulewidth}{0.4pt}

% Настройка кода
\definecolor{codegreen}{rgb}{0,0.6,0}
\definecolor{codegray}{rgb}{0.5,0.5,0.5}
\definecolor{codepurple}{rgb}{0.58,0,0.82}
\definecolor{backcolour}{rgb}{0.95,0.95,0.92}

\lstdefinestyle{mystyle}{
    backgroundcolor=\color{backcolour},
    commentstyle=\color{codegreen},
    keywordstyle=\color{magenta},
    numberstyle=\tiny\color{codegray},
    stringstyle=\color{codepurple},
    basicstyle=\ttfamily\footnotesize,
    breakatwhitespace=false,
    breaklines=true,
    captionpos=b,
    keepspaces=true,
    numbers=left,
    numbersep=5pt,
    showspaces=false,
    showstringspaces=false,
    showtabs=false,
    tabsize=2
}

\lstset{style=mystyle}

% Настройка отступов
\setlength{\parindent}{0pt}
\setlength{\parskip}{1\baselineskip}

\title{Полное руководство по Prism: AI-редактору \LaTeX}
\author{Дуплей Максим Игоревич}
\date{26 января 2026 г.}

\begin{document}

\maketitle
\tableofcontents
\newpage

\section{Введение в Prism}

\subsection{Что такое Prism?}

\textbf{Prism} --- это революционный \LaTeX{}-редактор, созданный OpenAI, который сочетает в себе мощь традиционного научного набора текста с возможностями искусственного интеллекта. Это не просто редактор --- это интеллектуальный помощник для написания научных документов, статей, диссертаций и других академических работ.

\subsection{Основные преимущества}

\begin{itemize}
    \item \textbf{Интеграция с ChatGPT:} Встроенный доступ к мощным возможностям GPT прямо в редакторе
    \item \textbf{Реальное время:} Совместная работа с соавторами в режиме реального времени
    \item \textbf{Интеллектуальные функции:} Автоматическая генерация контента, исправление ошибок, улучшение ясности
    \item \textbf{Проектный контекст:} ИИ имеет доступ ко всему вашему проекту для более точных ответов
    \item \textbf{Математическая поддержка:} Отличная поддержка сложных математических выражений
    \item \textbf{Unicode-совместимость:} Полная поддержка кириллицы и других языков
\end{itemize}

\section{Начало работы}

\subsection{Доступ к Prism}

Prism доступен по адресу: \url{https://prism.openai.com/}

\subsection{Создание нового проекта}

\begin{enumerate}
    \item Перейдите на сайт Prism
    \item Войдите в систему с помощью учетной записи OpenAI
    \item Нажмите "New Project"
    \item Выберите шаблон или начните с пустого документа
    \item Назовите ваш проект
\end{enumerate}

\subsection{Интерфейс редактора}

\begin{description}
    \item[Главное окно редактирования:] Центральная область для написания \LaTeX{}-кода
    \item[Панель ИИ:] Боковая панель для взаимодействия с ChatGPT
    \item[Предварительный просмотр:] Правая панель для просмотра рендеринга документа
    \item[Навигация по проекту:] Дерево файлов слева
    \item[Панель инструментов:] Верхняя панель с основными командами
\end{description}

\section{Основные возможности}

\subsection{Интеграция с ИИ}

\subsubsection{Базовые команды}

\begin{lstlisting}[language=TeX,caption=Примеры базовых команд ИИ]
% Добавление математических формул
"Добавь в введение формулу преобразования Лапласа для $t\cos(at)$."

% Создание таблиц
"В раздел резюме добавь таблицу 4×4."

% Редактирование текста
"Вычитай этот текст, отметь ошибки или логические пробелы и предложи, как улучшить ясность раздела."

% Анализ теорем
"Не пропустил ли я следствия или дальнейшие импликации Теоремы 3.1? Все ли оценки точные, или какие-то можно ослабить?"
\end{lstlisting}

\subsubsection{Генерация контента}

\begin{lstlisting}[language=TeX,caption=Генерация различных элементов]
% Аннотация
"Напиши аннотацию (abstract) на основе остальной части статьи."

% Библиография
"Добавь библиографию к статье и предложи связанные работы, которые я мог пропустить."

% Визуализации
"Сгенерируй эту нарисованную от руки схему в \LaTeX{}."

% Резюме
"Сгенерируй 200-словное резюме для широкой аудитории на немецком."

% Презентации
"Сгенерируй презентацию Beamer, где каждый слайд находится в отдельном файле."
\end{lstlisting}

\subsection{Математические возможности}

\subsubsection{Сложные формулы}

\[
\mathcal{L}\left\{ t \cos(a t) \right\} = \frac{ s^2 - a^2 }{ (s^2 + a^2)^2 }
\]

\[
\int_{-\infty}^{\infty} e^{-x^2} dx = \sqrt{\pi}
\]

\[
\sum_{n=1}^{\infty} \frac{1}{n^2} = \frac{\pi^2}{6}
\]

\subsubsection{Матрицы и системы}

\[
\begin{pmatrix}
a_{11} & a_{12} & \cdots & a_{1n} \\
a_{21} & a_{22} & \cdots & a_{2n} \\
\vdots & \vdots & \ddots & \vdots \\
a_{m1} & a_{m2} & \cdots & a_{mn}
\end{pmatrix}
\]

\[
\begin{cases}
x + 2y + 3z = 6 \\
2x - y + z = 3 \\
3x + y - 2z = 1
\end{cases}
\]

\subsection{Визуализации и диаграммы}

\subsubsection{Коммутативные диаграммы}

\begin{center}
\resizebox{0.6\linewidth}{!}{$
\begin{tikzcd}[row sep=2em, column sep=1.5em, ampersand replacement=\&]
E
  \arrow[dr, "e"']
  \arrow[drr, "p_2"]
  \arrow[ddr, "p_1"']
\& \& \\
\& A \times B \arrow[r, "\pi_2"'] \arrow[d, "\pi_1"] \& B \arrow[d, "g"] \\
\& A \arrow[r, "f"'] \& C
\end{tikzcd}
$}
\end{center}

\subsubsection{Таблицы}

\begin{center}
\resizebox{0.5\linewidth}{!}{%
\begin{tabular}{|c|c|c|c|}
\hline
1 & 2 & 3 & 4 \\
\hline
5 & 6 & 7 & 8 \\
\hline
9 & 10 & 11 & 12 \\
\hline
13 & 14 & 15 & 16 \\
\hline
\end{tabular}%
}
\end{center}

\section{Совместная работа}

\subsection{Приглашение соавторов}

\begin{enumerate}
    \item Откройте проект в Prism
    \item Нажмите меню "Share" в правом верхнем углу
    \item Введите email адреса соавторов
    \item Выберите уровень доступа (просмотр, комментирование, редактирование)
    \item Отправьте приглашения
\end{enumerate}

\subsection{Реальное время редактирования}

\begin{itemize}
    \item Все изменения синхронизируются в реальном времени
    \item Курсоры других пользователей отображаются разными цветами
    \item История изменений сохраняется автоматически
    \item Возможность отката к предыдущим версиям
\end{itemize}

\subsection{Комментарии и обсуждения}

\begin{enumerate}
    \item Выделите текст, к которому хотите добавить комментарий
    \item Нажмите "Leave a comment" или используйте горячие клавиши Ctrl+Alt+M
    \item Введите ваш комментарий
    \item Комментарии отображаются в боковой панели
    \item Соавторы могут отвечать на комментарии
\end{enumerate}

\section{Продвинутые возможности}

\subsection{Управление проектами}

\subsubsection{Структура проекта}

\begin{lstlisting}[language=TeX,caption=Пример структуры проекта]
project/
├── main.tex          % Главный файл
├── chapters/         % Главы
│   ├── introduction.tex
│   ├── methodology.tex
│   ├── results.tex
│   └── conclusion.tex
├── figures/          % Изображения
│   ├── diagram1.pdf
│   └── graph.png
├── bibliography.bib  % Библиография
└── styles/           % Стили и классы
    └── custom.sty
\end{lstlisting}

\subsubsection{Зависимости и компиляция}

\begin{lstlisting}[language=TeX,caption=Управление зависимостями]
% Prism автоматически отслеживает зависимости
% и предлагает добавить недостающие пакеты

"Добавь недостающие зависимости по всему проекту."
\end{lstlisting}

\subsection{Настройка и персонализация}

\subsubsection{Пользовательские шаблоны}

\begin{lstlisting}[language=TeX,caption=Пример пользовательского шаблона]
\documentclass[12pt]{article}
\usepackage[utf8]{inputenc}
\usepackage[russian]{babel}
\usepackage{amsmath,amssymb}
\usepackage{graphicx}
\usepackage{hyperref}

% Пользовательские команды
\newcommand{\R}{\mathbb{R}}
\newcommand{\N}{\mathbb{N}}
\newcommand{\Z}{\mathbb{Z}}

\title{Мой научный документ}
\author{Автор}
\date{\today}

\begin{document}
\maketitle
% Содержание документа
\end{document}
\end{lstlisting}

\subsubsection{Конфигурация компилятора}

\begin{lstlisting}[language=TeX,caption=Выбор компилятора]
% !TEX program = xelatex
% NOTE: Для корректного Unicode используйте XeLaTeX (или LuaLaTeX).

% Альтернативные варианты:
% !TEX program = pdflatex
% !TEX program = lualatex
\end{lstlisting}

\section{Решение проблем}

\subsection{Частые ошибки}

\subsubsection{Проблемы с компиляцией}

\begin{description}
    \item[Ошибка: "File not found"] Проверьте пути к файлам и зависимости
    \item[Ошибка: "Undefined control sequence"] Убедитесь, что все необходимые пакеты подключены
    \item[Ошибка: "Unicode character"] Используйте XeLaTeX или LuaLaTeX для Unicode-содержимого
\end{description}

\subsubsection{Проблемы с ИИ}

\begin{description}
    \item[Медленные ответы] Проверьте интернет-соединение
    \item[Неточности в ответах] Уточните запрос или предоставьте больше контекста
    \item[Ошибки в коде] Попросите ИИ проверить и исправить код
\end{description}

\subsection{Оптимизация производительности}

\begin{itemize}
    \item Используйте относительные пути к файлам
    \item Оптимизируйте изображения перед включением
    \item Разбивайте большие документы на части
    \item Регулярно очищайте временные файлы
\end{itemize}

\section{Лучшие практики}

\subsection{Организация проекта}

\begin{enumerate}
    \item Создавайте логическую структуру папок
    \item Используйте осмысленные имена файлов
    \item Поддерживайте регулярные резервные копии
    \item Документируйте сложные части проекта
\end{enumerate}

\subsection{Совместная работа}

\begin{enumerate}
    \item Установите четкие правила именования
    \item Используйте комментарии для объяснения сложных решений
    \item Регулярно синхронизируйте изменения
    \item Проводите периодические проверки кода
\end{enumerate}

\subsection{Использование ИИ}

\begin{enumerate}
    \item Будьте конкретны в запросах
    \item Предоставляйте достаточный контекст
    \item Проверяйте сгенерированный код
    \item Используйте ИИ для идей, но сохраняйте критическое мышление
\end{enumerate}

\section{Заключение}

Prism представляет собой мощный инструмент для научного письма, сочетающий традиционные возможности \LaTeX{} с современными технологиями искусственного интеллекта. Его интеграция с ChatGPT, поддержка совместной работы в реальном времени и интеллектуальные функции делают его идеальным выбором для написания сложных научных документов.

\subsection{Дальнейшие ресурсы}

\begin{itemize}
    \item Официальный сайт: \url{https://prism.openai.com/}
    \item Документация \LaTeX: \url{https://www.latex-project.org/help/documentation/}
    \item Математические символы: \url{https://oeis.org/wiki/List_of_LaTeX_mathematical_symbols}
    \item Сообщество \LaTeX: \url{https://tex.stackexchange.com/}
\end{itemize}

\subsection{Контактная информация}

\textbf{Автор:} Дуплей Максим Игоревич

\textbf{Telegram:} 
\begin{itemize}
    \item @quadd4rv1n7
    \item @dupley\_maxim\_1999
\end{itemize}

\textbf{Email:} maksimqwe42@mail.ru

\end{document}
